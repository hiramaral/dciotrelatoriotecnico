\chapter{INFORMAÇÕES GERAIS}
\label{cap:informações gerais}

\section{Objetivo Geral}
Desenvolver um sistema de monitoramento para coletar, processar e gerenciar informações dos níveis de temperatura, de umidade, de gás inflamável e de fumaça em Datacenters. A partir dos dados sensoriados (IoT), serão utilizadas técnicas de Inteligência Artificial e de Big Data que permitem melhorar e otimizar o funcionamento dos Datacenters.



\section {Escopo}

\subsection {Escopo Geral}

DCIoT detectará os eventos que não constem em seu banco de dados e detectará a concentração de vários gases combustíveis e fumaça em um ambiente. DCIoT será composto dos seguintes módulos:
- DCIoT SMART – é composto pelos componentes de hardware e de software embarcado. 
- DCIoT APP – aplicativo utilizado para configurar e monitorar a temperatura, a umidade, os níveis de gás inflamável e de fumaça no Datacenter.
- DCIoT Web – módulo de Inteligência Artificial hospedado na Web para processamentos dos dados.

\subsection {Escopo Específico}

DCIoT SMART
- Hardware - Dispositivo baseado em Samsung Artik com microcontrolador, sensores de temperatura, de umidade, de gás inflamável e de fumaça; buzzer, relés e outros dispositivos necessários para integração com o Smart Things. A fonte de energia será via USB, com conexão via Wi-Fi.
- Software – são formados pelos softwares embarcado no módulo de Hardware, sistema operacional, aplicativos necessários para o funcionamento do dispositivo, inteligência artificial, comunicação na web e integração em clouding
DCIoT APP – Aplicativo Android que será desenvolvido em REACT NATIVE para configurar, monitorar e utilizar a DCIoT. 
DCIoT Web – Aplicativo Web que será desenvolvido em REACT/FLASK para analisar e processar os dados enviados pelo Datacenter. Será também o aplicativo do 


administrador do sistema Datacenter. Haverá um sistema de dashboards e gráficos para dar apoio ao administrador.

\subsection {Escopo Negativo}

Não será desenvolvida uma versão IOS do aplicativo.


\section{DCIoT AMBIENTE DE HARDWARE}

Comente sobre o ambiente de hardware no PP4.0
\begin{figure}[h]
	\centering
	\includegraphics[width=15cm, height=10cm]{DCIOT ambiente de hardware.jpg} 
	\caption{Ambiente de hardware}
	\label{F1}
\end{figure}


%===================================================================
